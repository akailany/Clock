\documentclass{article}
\usepackage{fullpage}

\input{macros}

\setlength{\parskip}{.4cm}
\setlength{\baselineskip}{15pt}
\setlength{\parindent}{0cm}

\author{}
\date{}

\begin{document}

\hrule
\begin{center}
{\LARGE \sc $\underline{\mbox{Problem Set 6}}$} \\
{\Large \sc CS 373: Theory of Computation}\\
\ \\
\begin{tabular}{ll}
Assigned: October 10, 2013 & Due on: October 17, 2013
\end{tabular}
\end{center}
\hrule

{\bf Instructions:} This homework has 3 problems that can be solved in
groups of size at most 3. Please follow the homework guidelines given
on the class website. Solutions not following these guidelines will
not be graded.

{\bf Recommended Reading:} Lectures 12, and 13.

\begin{problem}{Design+Proof}
Let $L$ be the language consisting all strings over $\{a,b\}$ that
have twice as many $a$s as $b$s. For example, $aababa \in L$ and
$\emptystr \in L$ but $a \not\in L$.
\begin{enumerate}
\item Design a context-free grammar for $L$.\points{5}
\item Prove that your grammar is correct.\points{5}
\end{enumerate}
\end{problem}

\begin{problem}{Comprehension+Design}
Let $G = (V,\Sigma,R,\tuple{\mbox{STMT}})$ be the following grammar
\[
\begin{array}{rcl}
\tuple{\mbox{STMT}} & \longrightarrow & \tuple{\mbox{ASSIGN}}\: |\: 
  \tuple{\mbox{IF-THEN}}\: |\: \tuple{\mbox{IF-THEN-ELSE}}\\
\tuple{\mbox{IF-THEN}} & \longrightarrow & {\tt if\ condition\ then\ } 
  \tuple{\mbox{STMT}}\\
\tuple{\mbox{IF-THEN-ELSE}} & \longrightarrow & {\tt\ if\ condition\ then\ } 
  \tuple{\mbox{STMT}}\ {\tt else\ } \tuple{\mbox{STMT}}\\
\tuple{\mbox{assign}} & \longrightarrow & {\tt a := 1}
\end{array}
\]
where $\Sigma = \{{\tt if}, {\tt then}, {\tt else}, {\tt condition},
{\tt a := 1}\}$ and $V =
\{\tuple{\mbox{STMT}},\tuple{\mbox{IF-THEN}},\tuple{\mbox{IF-THEN-ELSE}},
\tuple{\mbox{ASSIGN}}\}$. $G$ is a natural looking grammar for a
fragment of a programming language, but $G$ is ambiguous.
\begin{enumerate}
\item Show that $G$ is ambiguous.\points{5}
\item Give a new unambiguous grammar for the same language. You need
  not prove that your grammar is correct but explain your
  construction. You may want to look at examples in Lecture
  12.\points{5}
\end{enumerate}
\end{problem}

\begin{problem}{Design}
Design a PDA to recognize the language $C = \{x\# y\: |\: x,y \in
\{0,1\}^* \mbox{ and } x \neq y\}$; thus, $C \subseteq
\{0,1,\#\}^*$. You need not prove that your construction is correct, but
you should clearly explain the intuition behind your
construction.\points{10}
\end{problem}

\end{document}
